% Created 2019-06-27 木 15:54
% Intended LaTeX compiler: pdflatex
\documentclass{article}

\usepackage{authblk}
\usepackage[dvipdfmx]{graphicx}
\usepackage{url}
\usepackage{amsmath}
\usepackage{amssymb}

\date{}
\title{A decision diagram-based solver for the independent set reconfiguration problem}
\pagenumbering{gobble}

\author[1]{Jun Kawahara}
\author[1]{Hiroki Yamazaki}
%\author[1]{zzz}
\affil[1]{Kyoto University}

\begin{document}
\maketitle

\section{Engine: Core Solver or Algorithm}
Our solver is based on the algorithm~\cite{Ito2022} that uses
zero-suppressed binary decision diagrams~\cite{Minato93} and the graphillion library~\cite{Inoue2016}.

\section{ZDD}

A ZDD is a data structure for efficiently representing a family of sets.
The definition of a ZDD is given below.
Let $X = \{x_1, \ldots,x_n\}$ be an underlying set of a family represented by a ZDD,
and let $x_1 < x_2 < \cdots < x_n$.
A ZDD is a directed acyclic graph (DAG) $D = (N, A)$ with the following property.
$D$ has at most two nodes with outdegree zero, which are called 0-terminal and 1-terminal.
Nodes other than the terminals are called non-terminal nodes. A non-terminal node has one element of $X$ and is called a label.
A non-terminal node has two arcs, called a 0-arc and a 1-arc.
If a non-terminal node $\nu \in N$ has label $x \in X$ and its 0-arc and 1-arc point at $\nu_0, \nu_1 \in N$, respectively,
we write $\nu = (x, \nu_0, \nu_1)$.
If $\nu_i = (x^i, \nu_{i0}, \nu_{i1})$ holds, $x < x^i$ must be satisfied ($i = 0, 1$).
$D$ has exactly one node with indegree zero, called the root.

A node $\nu$ in a ZDD recursively represents a family of sets as follows.
If $\nu$ is the 0- and 1-terminal, it represents $\emptyset$ and $\{\emptyset\}$, respectively.
If $\nu$ is a non-terminal node, we write $\nu = (x, \nu_0, \nu_1)$.
The family $\mathcal{S}(\nu)$ of sets is the union of the family of sets represented by $\nu_0$
and the family of sets obtained by adding $x$ to each element of the family of sets represented by $\nu_1$.
That is, $\mathcal{S}(\nu) = \mathcal{S}(\nu_0) \cup (\{\{x\}\} \Join \mathcal{S}(\nu_1))$,
where $\mathcal{F} \Join \mathcal{G} = \{F \cup G \mid F \in \mathcal{F},
G \in \mathcal{G} \}$.
We interpret the family of sets represented by its ZDD as the family of sets represented by the root node.
For more information on ZDDs, please refer to \cite{Knuth2011}.

\section{Algorithm for solving the independent set reconfiguration problem}

Given a graph $G = (V, E)$, an independent set of $G$ can be represented as a subset of $V$.
The collection of all independent sets of $G$ can be represented by a family of sets with underlying set $V$.
For a family $\mathcal{F}$ of sets with underlying set $V$ and a set $A \in V$,
let $\mathsf{swap}(\mathcal{F}, A)$ be the family of all the sets obtained by
removing an element from every set in $\mathcal{F}$
and adding an element in $A$ to the set. That is,
\begin{equation*}
  \mathsf{swap}(\mathcal{F}, A) = \{ F \cup \{x\} \setminus \{x'\}
      \mid F \in \mathcal{F},
      %\quad\quad\quad
      x \notin F, x \in A, x' \in F \}.
\end{equation*}

Our algorithm for computing $\mathsf{swap}(\mathcal{F}, A)$ is described in the next section.
Using $\mathsf{swap}$, the token-jumping model of the independent set reconfiguration problem can be solved as follows.
In what follows, the underlying set of all families is $V$.
Let $\mathcal{F}_{\mathrm{ind}}$ be the family of all the independent sets of $G$.
Let $\mathcal{F}_0 = \{S\}$.
We recursively define $\mathcal{F}_{i + 1} = \mathsf{swap}(\mathcal{F}_i, V) \cap \mathcal{F}_{\mathrm{ind}}$.
$\mathsf{swap}(\mathcal{F}_i, V)$ is the family of sets obtained by adding and removing one element of each set of $\mathcal{F}_i$.
By taking the intersection of $\mathsf{swap}(\mathcal{F}_i, V)$ and $\mathcal{F}_{\mathrm{ind}}$, 
we extract independent sets from $\mathsf{swap}(\mathcal{F}_i, V)$.
$\mathcal{F}_{i}$ is the family of independent sets obtained by reconfiguring $S$ using at most $i$ steps
(satisfying that the intermediate sets are also independent sets).
The above computation is performed for $i = 0, 1,\ldots$ in this order.
If $T \in \mathcal{F}_{i}$ holds for some $i$,
we know that there exists a reconfiguration sequence from $S$ to $T$ with length $i$, and the algorithm stops.
If $\mathcal{F}_{i} = \mathcal{F}_{i + 1}$ for some $i$,
no new independent set is obtained, and we know that there is no reconfiguration sequence from $S$ to $T$.
From the construction of $\mathcal{F}_{i}$,
the smallest $i$ such that $T \in \mathcal{F}_{i}$ holds is the minimum number of steps
among all the reconfiguration sequences from $S$ to $T$.

\section{Reconfiguration operation for families of sets using ZDDs}

In this section, we describe how to realize the algorithm described in the previous section using ZDDs.
A method to construct a ZDD representing $\mathcal{F}_{\mathrm{ind}}$ is described in the book \cite{Knuth2011}.
The construction of a ZDD for $\mathcal{F}_0 =\{S\}$ is obvious.
For any set $A$, there exists a method to determine whether $A$ belongs to $\mathcal{F}$.
Given two families $\mathcal{F}, \mathcal{F}'$ of sets as ZDDs,
a method for constructing a ZDD representing $\mathcal{F} \cap \mathcal{F}'$ has been proposed~\cite{Minato93}.
Therefore, if there is a method for constructing a ZDD representing $\mathsf{swap}(\mathcal{F}, A)$,
all the above procedures can be performed as ZDD operations.

Given a family $\mathcal{F}$ of sets as a ZDD,
we describe how to construct a ZDD representing $\mathsf{swap}(\mathcal{F}, A)$.
The $\mathsf{swap}$ operation uses the following operations:
\begin{align*}
  \mathsf{rem}(\mathcal{F}) & =  \{ F \setminus \{x\}
  \mid F \in \mathcal{F}, x \in F \}, \\
  \mathsf{add}(\mathcal{F}, A) & = \{ F \cup \{x\} 
      \mid F \in \mathcal{F}, x \notin F, x \in A \}.
\end{align*}
For every $x \in X$,
we can write $\mathcal{F} = \mathcal{F}_0 \cup (\{\{x\}\} \Join \mathcal{F}_1)$.
We let $\mathcal{F}^{\mathrm{rem}} = \mathsf{rem}(\mathcal{F})$ and
$\mathcal{F}^{\mathrm{rem}} = \mathcal{F}_0^{\mathrm{rem}} \cup (\{\{x\}\} \Join \mathcal{F}_1^{\mathrm{rem}})$,
where any set in $\mathcal{F}_0^{\mathrm{rem}}$ and $\mathcal{F}_1^{\mathrm{rem}}$ does not contain $x$.

Each element of $\{\{x\}\} \Join \mathcal{F}_1^{\mathrm{rem}}$ is obtained by removing one element other than $x$
from each set in $\{\{x\}\} \Join \mathcal{F}_1$. Thus, $\mathcal{F}_1^{\mathrm{rem}} = \mathsf{rem}(\mathcal{F}_1)$ holds.
Also, each element of $\mathcal{F}_0^{\mathrm{rem}}$ is obtained by
removing one element from each set in $\mathcal{F}_0$, or by removing $x$ from each set in $\{\{x\}\} \Join \mathcal{F}_1$.
Thus, $\mathcal{F}_0^{\mathrm{rem}} = \mathsf{rem}(\mathcal{F}_0) \cup \mathcal{F}_1$ holds.


We describe our algorithm $\mathsf{rem}(\nu)$ that constructs a ZDD representing the family $\mathsf{rem}(\mathcal{S}(\nu))$ of sets
for a ZDD $\nu = (x, \nu_0, \nu_1)$.
If $\nu$ is the 0- or 1-terminal, the 0-terminal is returned. Otherwise, we perform the following.
First, recursively compute $\mathsf{rem}(\nu_0)$ and take the union of it and $\nu_1$ (as a ZDD operation)~\cite{Knuth2011}.
Next, recursively compute $\mathsf{rem}(\nu_1)$.
Finally, return $(x, \mathsf{rem}(\nu_0) \cup \nu_1, \mathsf{rem}(\nu_1))$ as the root node of $\mathsf{rem}(\nu)$.

For a ZDD $\nu = (x, \nu_0, \nu_1)$,
the algorithm $\mathsf{add}(\nu, A)$, which constructs a ZDD representing $\mathsf{add}(\mathcal{S}(\nu), A)$,
can be written as well as $\mathsf{rem}$ in a recursive mannar.
If $\nu$ is the 0-terminal, it returns the 0-terminal.
If $\nu$ is the 1-terminal, it returns the ZDD representing $\{\{x\} \mid x \in A\}$
(the construction of the ZDD is clear).
If $x < \min(A)$, return $(x, \mathsf{add}(\nu_0, A), \mathsf{add}(\nu_1, A))$
because $x < \min(A)$ means $x \notin A$,
where $\min(A)$ is the element $y$ in $A$ such that for all $y' (\neq y) \in A$, $y < y'$ holds. 
If $x = \min(A)$, return $(x, \nu'_0, \nu'_1)$,
where $\nu'_0 = \mathsf{add}(\nu_0, A \setminus \{x\})$,
and $\nu'_1 = \mathsf{add}(\nu_1, A \setminus \{x\}) \cup \nu_0$.
If $x > \min(A)$, return $(\min(A), \mathsf{add}(\nu, A \setminus \{x\}), \nu)$.

For a ZDD $\nu = (x, \nu_0, \nu_1)$,
the algorithm $\mathsf{swap}(\nu, A)$, which constructs a ZDD representing $\mathsf{swap}(\mathcal{S}(\nu), A)$,
can be written as well as $\mathsf{rem}$ and $\mathsf{add}$ in a recursive mannar.
If $\nu$ is the 0- or 1-terminal, it returns the 0-terminal.
If $x < \min(A)$, return $(x, \mathsf{swap}(\nu_0, A), \mathsf{swap}(\nu_1, A))$.
If $x = \min(A)$, return $(x, \nu'_0, \nu'_1)$,
where $\nu'_0 = \mathsf{swap}(\nu_0, A \setminus \{x\}) \cup \mathsf{add}(\nu_1, A \setminus \{x\})$,
and $\nu'_1 = \mathsf{swap}(\nu_1, A \setminus \{x\}) \cup \mathsf{rem}(\nu_0)$.
If $x > \min(A)$, return $(\min(A), \mathsf{swap}(\nu, A \setminus \{x\}), \mathsf{rem}(\nu))$.

%\section{Computation Environment}
%\begin{itemize}
%\item CPU: xxx
%\item MEM: yyy
%\item zzz etc
%\end{itemize}

\begin{thebibliography}{99}

\bibitem{Inoue2016}
  Inoue, T., Iwashita, H., Kawahara, J., Minato, S.,
  \newblock Graphillion: software library
    for very large sets of labeled graphs, International Journal on Software
    Tools for Technology Transfer 18~(1) 57--66, 2016.
  \newblock \url {https://doi.org/10.1007/s10009-014-0352-z}

\bibitem{Ito2022}
  Ito, T., Kawahara, J., Nakahata, Y., Soh, T., Suzuki, A., Teruyama, J., Toda, T.,
  \newblock ZDD-based algorithmic framework for solving shortest reconfiguration problems,
  \newblock arXiv:2207.13959v1, 2022.

\bibitem{Knuth2011}
  Knuth, D.~E.,
  \newblock {The Art of Computer Programming, Volume 4A, Combinatorial
  Algorithms, Part 1}, 1st Edition, Addison-Wesley Professional, 2011.


\bibitem{Minato93}
Minato, S.,
\newblock Zero-suppressed {BDD}s for set manipulation in combinatorial
  problems.
\newblock In {\em the 30th ACM/IEEE design automation conference}, pages
  272--277, 1993.
\newblock \url {https://doi.org/10.1145/157485.164890}



\end{thebibliography}

\end{document}
